


% Write something about machine unlearning in general
Before exactly defining machine unlearning, the general purpose of machine learning will be described first. Given a dataset $\mathcal{D}$, the objective is with some training algorithm $\mathcal{T}$ to learn a model $\mathcal{M}$ to perform as good as possible 



Tradeoff between
\begin{enumerate}
    \item The forgetness
    \item the efficiency
    \item high accuracy after forgetness
\end{enumerate}




















\[
    \mathcal{D}_{te} \cap \mathcal{D}_{tr} = \mathcal{D}
\]

\[
    \mathcal{D}_{te} \cup \mathcal{D}_{tr} = \emptyset
\]


machine unlearning of zann because they genreally are compututional heavty





Machine Unlearning can be separated into different categories \textit{Weak Unlearning}, \textit{Strong Unlearning} and \textit{Exact Unlearning}.


 

\begin{definition}[Weak Unlearning]
    here we only try to keep the output of the two models exactly the same (and do not consider all the weights inside the model)
\end{definition}

APPROXIMATIVE LEARNING captures a smaller statement. Here one wants to keep the KL between the distributions within a certain tolerance. This can be classified into two smaller catagories.

\begin{definition}[Strong Unlearning]
     Here the goal is for all the weights to be approximately the same
\end{definition}

\begin{definition}[Exact Unlearning]
    dump
\end{definition}

The models hold information and therefore it is not enough to just remove the data from the dataset. 





\[
    \mathcal{K}(P(\mathcal{U}(\mathcal{A}(\mathcal{D}),\mathcal{D},\mathcal{D}_u)), 
    P(\mathcal{A}(\mathcal{D}\backslash\mathcal{D}_u)))=0
\]
Where $\mathcal{A}$ is the learning process, $\mathcal{U}$ is the unlearning process, $\mathcal{K}$ is the KL divergence, $\mathcal{D}$ is the total dataset, $\mathcal{D}_u$ is the wished data to be unlearned. $P$ signifies the distribution of the weights.





EVALUATION (Verification)
\begin{itemize}
    \item Retraining based verification (traning a model from bottom and comparing)
    \item Tryning to simulate an attack
    \item Relearning time-based verification (does the model improve when learned on the unlearned data? If yes, then the model probably still captures some information).
\end{itemize}



Two classes of unlearning methods: \textbf{Data Reorganisation} and \textbf{Model Manipulation}.

\textbf{Data Reorganisation}
\begin{itemize}
    \item Data obfuscation: Intentionally adding some choreographed data, as to fine tune the model with such that it unlearns the information. This could be done by adding the $\mathcal{D}_u$ but with randomly selected incorrect labels.
    \item Data pruning
    \item Data replacement
\end{itemize}

\textbf{Model Manipulation}
\begin{itemize}
    \item Model shifting
    \item Model replacement 
    \item Model pruning
\end{itemize}






\section{Data Reorganisation}






\section{Model Manipulation}






\section{Amnesiac Unlearning}

The main goal of Amnesiac Unlearning as described in \cite{graves_amnesiac_2020} is to remember each time an observation is used to update parameters. Specifically, this includes remembering each batch that the certain observation was a part of among with each of the parameters updates. Suppose one starts with some  model with initial parameters $\theta_{initial}$. Then to obtain some model $\theta_{M}$, the initial parameters are updated for $E$ epochs containing $B$ batches each. Then the updates of the parameters can be described as:
\[
    \theta_M = \theta_{initial} + \sum^E_{e=1} \sum^B_{b=1} \Delta_{\theta_{e,b}}
\]
Thereby obtaining a trained model $M$. Now suppose some observations are to be removed. Then the set of \textit{sensitive batches}: $SB$ are used to remove the learned data from the model. This can be formulated as:
\[
    \theta_{M'} = \theta_{initial} + \sum^E_{e=1} \sum^B_{b=1} \Delta_{\theta_{e,b}} - \sum_{sb=1}^{SB} \Delta_{\theta_{sb}}
\]
This could of course fast become infeasible due to the need to store a big amount of parameter updates, having a space complexity of $\mathcal{O}(EB)$.




\section{SCRUB}
SCalable Remembering and Unlearning unBound



Having a teacher-student system, where the certain samples are learned to the student to be bad at predciting it. 

\url{https://proceedings.neurips.cc/paper_files/paper/2023/file/062d711fb777322e2152435459e6e9d9-Paper-Conference.pdf}




The following is from \cite{xu_machine_2023}.

Machine unlearning to avoiud membership inference attacks (\textbf{check out [5]}) and model inversion attacks (check out [6])

\textbf{membership inference attacks}

Given a data record and black-box access to a model, determine if the record was in the model's training data. 

Information leakage.

Motivation: Model behaves differently on data that is has been trained on vs data it has not. This is often dispicted as overfitting but can also be other things. The model itself has an importance for how much the information is leaked as shown in \cite{shokri_membership_2017}. 

By black-box it is meant that there is acess to the prediction of the model, but not the actual weights of the model. 




\textbf{model-inversion}


\cite{fredrikson_model_2015}




